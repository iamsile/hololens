Abstract:
	Ikea, the world's largest furniture retailer maintains multiple commerce channels for consumers
in print, online, and in stores with ~200 million printed catalogues, >700 million store visits,
and >2 billion visits to ikea.com. The largest of the three, e-commerce yielding a rate of returns
of ~10 percent, three times higher than that of transactions made at brick-and-mortar stores. This
difference is attributed to the consumer being less informed about the product at the time of
purchase when shopping for furniture online, leading to dissatisfaction with the size, aesthetics
or decorum compatibility with the furniture post-transaction.
	This project utilizes augmented reality software on the Hololens platform to provide better spatial
and aesthetic information about the retail products in the Ikea catalogue, making remote e-commerce
on parity with in-store visits and decrease information asymmetry. This provides users a more
immersive experience and convenient relays spatial information before transactions without the need
for premature shipment, assembly or furniture movement/placement, ultimately reducing the rate of
returns/exchanges.
	The user's interactions in selecting, moving and pinning the holographic 3D models will be observed to
determine the efficacy of the software in assisting the consumer in selecting the most suitable product.
This is measured by the percent of users in the study able to identify the appropriate furniture to purchase
using the software versus looking at static images and text. The ease of use and perusal time for user's
interacting with the software is also measured to evaluate likelihood of adoption.
